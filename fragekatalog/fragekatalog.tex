\documentclass[hidelinks]{article}
% Seitengröße und Ränder
\usepackage[a4paper]{geometry}
\usepackage[utf8]{inputenc}
\usepackage[ngerman]{babel}
\usepackage{graphicx}
\usepackage{hyperref}
\usepackage{wrapfig}
% Keine Einrückung am Paragraphenanfang
\usepackage{parskip}
% Anführungszeichen
\usepackage{csquotes}
\MakeOuterQuote{"}

\title{Rechnernetze Uni Bremen Fragekatalog}
\begin{document}
\maketitle
\section{Digitale Übertragung}
\begin{enumerate}
\item \textit{Was versteht man unter digitaler Übertragungstechnik?}

Die Technik, die, in Abhängigkeit von Bittakt, Symbolrate, Übertragungsmedium etc., digitale Signale von A nach B transportiert. %TODO 'etc.'

\item \textit{Skizziere einige Übertragungsverfahren zur digitale Übertragung von Bitfolgen.}

Siehe Manchestercode, NRZI etc. %TODO Zeichnung einfügen
\item \textit{Was versteht man unter dem Begriff Bitsynchronisation?}

Bei der Übertragung kommt es zu einer Verzögerung zwischen Absenden und Empfangen. Um ein korrektes Übertragen von digitale Signalen zu ermöglichen, muss gewährleistet sein, dass beide Parteien die selbe Taktrate verwenden. Die Synchronisation dieser Raten bezeichnet man als Bitsynchronisation.
\item \textit{Welche grundsätzlich verschiedenen Verfahren zur Taktgewinnung können unterschieden werden?}

\begin{itemize}
	\item Nutzen einer seperaten Taktleitung, die abwechselnd im Takt hi und low sendet 
	\begin{itemize}
		\item Teuer
		\item Die beiden Leitungen müssten exakt gleich lang sein
	\end{itemize}
	\item Kodieren von Bits über Flanken (z.B. Manchester-Code)
	\begin{itemize}
		\item Sehr hektisch
	\end{itemize}
	\item Verwenden von lokalen Uhren
	\begin{itemize}
		\item Regelmäßige Flanken müssen sichergestellt werden (z.B. durch NRZI-Code und/oder Synchronisationsbits)
	\end{itemize}
\end{itemize}
\item \textit{Welche verschiedenen Gleichlaufverfahren gibt es?}

\begin{itemize}
	\item Asynchroner Betrieb
	\begin{itemize}
		\item Takt für jedes Byte neu aufsetzen
		\item Unterscheidung zwischen Daten und "Pause" damit implizit		
	\end{itemize}	
	\newpage
	\item Synchroner Betrieb
	\begin{itemize}
		\item Keine implizite Unterscheidung zwischen Daten und "Pause"
	\end{itemize}
\end{itemize}

\item \textit{Welche Probleme müssen synchrone Verfahren ohne Taktleitung lösen?\\ Beschreibe Lösungsalternativen.}

Im Gegensatz zu asynchronen Verfahren besteht bei synchronen Verfahren keine implizite Trennung von Datenübertragung und Pause. Dieses Problem kann z.B. dadurch gelöst werden, dass vor und nach der Übertragung von Daten ein Byte mit nur 1en gesendet wird. Vor und nach diesen Start- und Stop-Bits werden dann nur 0en gesendet. Anhand dieser Bits lassen sich Daten und Pause trennen und es besteht eine Möglichkeit zum Nachstellen lokaler Uhren.
\end{enumerate}
\section{Analoge Übertragung}
\begin{enumerate}
\setcounter{enumi}{6}
\item \textit{Was versteht man unter analoger Übertragungstechnik? Warum wird sie eingesetzt?}

Analoge Signale sind Schwingungen, welche zwar in einfachen Formen technisch veraltet sind, in komplexeren Formen (z.B. im Telefon- oder Funknetz) aber weiterhin von Bedeutung sind.

\item \textit{Was ist ein Modem?}

Ein Modem wandelt digitale in analoge Signale und analoge in digitale Signale um. Sie werden benutzt, um digitale Signale für die Übertragung auf analogen Netzen aufzubereiten.

\item \textit{Erläutere kurz die Grundzüge der Frequenz-, Amplituden- und Phasenmodulation.}

\begin{itemize}
\item Frequenzmodulation: Frequenz einer Schwingung entscheidet über Wert des Bits (hoch $\Rightarrow$ 1, niedrig $\Rightarrow$ 0)
\item Amplitudenmodulation: Ausschlag einer Schwingung entscheidet über Wert des Bits (hoch $\Rightarrow$ 1, flach $\Rightarrow$ 0)
\item Phasenmodulation: Beginn der Phase des Signals entscheidet über kodierten Bitwert (Visualisierung in Koordinatensystemen)
\end{itemize}

\item \textit{Wo ist der Unterschied zwischen den Maßeinheiten bit/s und baud?}

Die Einheit \textit{baud} korrespondiert mit der Signal- und damit auch der Taktrate, währen die Einheit \textit{bit} mit der Übertragungsrate korrespondiert.

\item \textit{Welche Bedeutung haben die Bezeichnungen "duplex", "halbduplex" und "simplex" bezogen auf ein physikalisches Übertragungsmedium?}

\begin{itemize}
\item Duplex $\Rightarrow$ Es können beide Richtungen gleichzeitig genutzt werden
\item Halbduplex $\Rightarrow$ Es können beide Richtungen abwechselnd genutzt werden
\item Simplex $\Rightarrow$ Es kann nur eine Richtung genutzt werden
\end{itemize}

\item \textit{Wie arbeiten Time Division Multiplexing (TDM), Frequency Division Multiplexing (FDM) und Echokompensation in etwa?}

\begin{itemize}
\item TDM $\Rightarrow$ Die Kommunikationsrichtung auf dem Medium wechselt in festgelegten zeitlichen Abschnitten
\item FDM $\Rightarrow$ Verwendung verschiedener Frequenzen (vgl. Radiofunk)
\item Echokompensation $\Rightarrow$ Verwendung der gleichen Frequenzen zum gleichen Zeitpunkt. Durch Kenntnis der eigenen gesendeten Daten können Überlagerungen wieder rausgerechnet werden
\end{itemize}

\item \textit{Warum sind die über ein Kommunikationsmedium ausgetauschten Signale nicht nur Nutzinformationen? Nenne Beispiele.}

Es müssen auch Steuerinformationen wie z.B. der Takt, eine Unterscheidung zwischen Daten und Pause oder Richtungsumschaltungsnachrichten übertragen werden.\\

\item \textit{Was ist ein Protokoll in diesem Zusammenhang?}

Es regelt Steuerinformationen und somit die Absprache zwischen den beiden Kommunikationspartnern.

\item \textit{Warum ist die Entwicklung von Standards im Bereich der Kommunikationssysteme so wichtig?}

Da sonst Kommunikation zwischen Geräten verschiedener Hersteller nicht gewährleistet werden kann.

\item \textit{Mit welchen internationalen Standardisierungsorganisationen hat man es hier hauptsächlich zu tun?}

\begin{itemize}
\item ISO (International Organization for Standardization; ISO xxxxx)
\item ITU-TSS (International Telecommunications Union - Telecommunication Standards Sector; V.xxx, X.xxx, I.xxx)
\end{itemize}
\end{enumerate}
\section{POTS und Modemanschluss}
\begin{enumerate}
\setcounter{enumi}{16}
\item \textit{Welche Arten von Festlegungen müssen bei einer physischen Kommunikationsschnittstelle	(wie z.B. der Modemschnittstelle) getroffen werden?}

\begin{itemize}
\item Elektrische Charakteristika
\item Bedeutung der Schnittstellenleitungen
\item Steckerform
\item Protokollverhalten
\end{itemize}

\item \textit{Wie läuft ein verbindungsorientiertes Protokoll im Grundsatz ab?}

\begin{itemize}
\item Ruhepause
\item Aufbau einer Kommunikationsbeziehung
\item Übertragung von Nutzinformationen
\item Abbau der Kommunikationsbeziehung
\end{itemize}

\item \textit{Was versteht man (in Zusammenhang mit verbindungsorientierten Protokollen) unter einer End-zu-End-Signifikanz des Verbindungsaufbaus bzw. unter einer Aufbaukollision?}

\begin{itemize}
\item End-zu-End-Signifikanz: %TODO
\item Aufbaukollision: Ein Endsystem A möchte ein Verbindung mit B aufbauen. Zeitgleich bekommt A eine Anfrage von C. Jetzt kann C Daten an A senden, während A noch versucht, eine Verbindung mit B aufzubauen
\end{itemize}

\item \textit{Wie können Wahlinformationen für den Verbindungsaufbau über eine V.24-Schnittstelle übertragen werden?}

\begin{itemize}
\item Impulswahlverfahren (Unterbrechung des Gleitstromkreises)
\item Mehrfrequenzwahlverfahren (Tonwahl, Kombination zweier Frequenzen auf dem Tastenfeld)
\end{itemize}

\item \textit{Was versteht man unter dem Begriff Bytesynchronisation?}

Es werden ein Sender und ein Empfänger auf eine solche Art und Weise synchronisiert, dass beide Parteien die gleiche Datenrate verwenden, der Empfänger die gesendeten Bytes wieder als solche trennen kann und die Signale damit richtig interpretieren kann

\item \textit{Auf welche Arten kann die Bytesynchronisation vorgenommen werden? Was ist Datentransparenz?}

\begin{itemize}
\item Asynchrone Bytesynchronisation: Implizite Bytesynchronisation, da jedes Byte durch Startbit idenzifiziert wird; Zähler bis 8, danach Stopbit(s)
\item Synchrone Bytesynchronisation: Im laufenden Betrieb Zähler bis 8. Problem: Anfangssynchronisation:
\begin{itemize}
	\item Leitung für Bittakt (teuer)
	\item Spezielles Bytesynchronisationsmuster auf der Datenleitung:
	\begin{itemize}
		\item ASCII: SYN-Zeichen
		\item In Aufbauphasen bzw. am Nachrichtenanfang unproblematisch
		\item in ASCII klar von Nutzinformationen getrennt
		\item Aber: Nicht an beliebiger Stelle für beliebige Bitströme nutzbar $\Rightarrow$ Datentransparenz $\Rightarrow$ Lösung: HDLC-Flag (Klassisch wird asynchron genutzt)
	\end{itemize}
\end{itemize}
\end{itemize}

\item \textit{Auf welche grundsätzlich verschiedenen Arten können in einem Protokoll Steuerinformationen von Nutzinformationen unterschieden werden? Nenne Beispiele ihrer Anwendung.}

\begin{itemize}
\item Hayes-Code (zunächst proprietär) $\Rightarrow$ ADTD-Code *Nummer* (Attention Dial Tone Code)
\item V.25bis $\Rightarrow$ Ähnlich, aber inkompatibel; hat sich nicht durchgesetzt
\item V.25ter $\equiv$ Hayes-Code $\Rightarrow$ sehr vielfältig, auch Aufbaue, Abbau und Optionsaushandlung
\end{itemize}
\end{enumerate}
\section{Architekturmodelle -- OSI vs. Internet}
\begin{enumerate}
\setcounter{enumi}{23}
\item \textit{Wie laufen Kommunikationsbeziehungen über eine Hierarchie von Protokollen ab?}

Protokolle höherer Schichten sehen den Dienst der unteren Schichten als Blackbox an und bauen darauf auf. Die einzige tatsächlich Kommunikation findet stets zwischen den beiden niedrigsten Schichten statt.

\item \textit{Erkläre den Begriff des Dienstes einer Schicht. Wie grenzt er sich gegenüber dem Protokoll-Begriff ab?}

Der Dienst einer Schicht ist die Leistung, die sie der jeweils höheren Schicht bereitstellt. Ein Protokoll auf der Schicht ist eine spezielle Art und Weise, diese Leistung zu erbringen.

\item \textit{Vergleiche die Architekturmodelle von OSI und Internet. Wo sind Gemeinsamkeiten und	Unterschiede?} 

Im OSI-Modell gibt es 7 Schichten, im Internet-Modell nur 4 (keine Presentation, Session, Subnetzlayer zusammengefasst). Außerdem folgten bei OSI die Protokolle auf das Modell, beim Internet-Modell andersrum.
\end{enumerate}
\section{Abschnittsicherungsschicht -- Konzepte}
\begin{enumerate}
\setcounter{enumi}{26}
\item \textit{Welche Aufgaben hat die Abschnittsicherungssicht}
\begin{itemize}
\item Geben eines Bytetaktes
\item Bewahren von Informationseinheiten (Frames)
\item Fehlersichere Übertragung
\item Schutz vor Überlastung
\item In der Regel Datentransparenz
\item Gegebenfalls Medienzugang
\end{itemize}
\item \textit{Wie arbeitet das CRC-Verfahren in etwa?}

Mit einem Generatorpolynom des Grades n wird die zu sendende Nachricht um n Nullen erweitert. Diese Nullen werden dann mittels XOR-Verknüpfungen zu Prüfbits abgewandelt. Durch "Division" der Nachricht durch das Generatorpolynoms kann der Empfänger jetzt feststellen, ob es Übertragungsfehler gab.

\item \textit{Wie arbeiten klassische Fehlerbehebungsverfahren?}

Der Empfänger kann nur feststellen, dass es einen Übertragungsfehler gab, diesen aber nicht beheben. Der Empfänger muss also beim Sender anfragen, die Daten nochmal zu erhalten. Nach so einer Anfrage oder einem Timeout sendet der Sender die Daten dann erneut. Die bereits gesendet Daten müssen beim Sender also bis zu einer Empfangsbestätigung gepuffert werden.

\item \textit{Was ist Flusskontrolle? Wie arbeiten klassische Flusskontrollverfahren?}

Flusskontrolle ist Überlastschutz für den Empfänger.
\begin{itemize}
\item Explizit $\Rightarrow$ Explizites Bremsen, Senden eines "Stop" statt "o.k.", Aufheben mit "Go"
\item Implizit $\Rightarrow$ Herauszögern der Bestätigung, sobald der Puffer voll ist, kann nicht mehr gesendet werden
\end{itemize}
\end{enumerate}

\section{Abschnittsicherungsschicht -- LAPB}

\section{Netztypen- und Topologien}

\section{Klassisches Ethnernet vs. FDDI}

\section{Übertragungsmedien -- Kupfer vs. Glas}

\section{Lokale Netze -- Ethernet heute}

\section{Netzkopplung}

\section{IPv4 und IPv6}

\section{Rund um IP}

\section{Routing}

\section{Lokale Funknetze -- WLAN vs. Bluetooth}

\section{Heimanschlüsse und -netze}

\section{Transportschicht -- Überblick}

\section{Transportschicht -- UDP, TCP, SCTP}

\section{Kommunikationssteuerung}

\section{Binäre Kodierungsformate -- ASN.1, XDR, Protobuf etc.}

\section{Offene Dokumentverarbeitung}

\section{Textuelle Kodierungsformate -- SGML, XML, JSON etc.}

\section{Prozedurfernaufruf}

\section{Namensdienste -- X.500, LDAP und DNS}

\section{Betriebsprotokolle -- SNMP, dhcp und NTP}

\section{Die drei Säulen des Web -- URI, HTML, HTTP}

\section{Web -- Komponenten und Services}

\section{Klassische Internetanwendungen -- Telnet, FTP und NFS}

\section{E-Mail}

\section{Informationssicherheit}

\section{Internet of Things (IoT)}
\end{document}
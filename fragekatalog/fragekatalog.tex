\documentclass[hidelinks]{article}
% Seitengröße und Ränder
\usepackage[a4paper]{geometry}
\usepackage[utf8]{inputenc}
\usepackage[ngerman]{babel}
\usepackage{graphicx}
\usepackage{hyperref}
\usepackage{wrapfig}
% Keine Einrückung am Paragraphenanfang
\usepackage{parskip}
% Anführungszeichen
\usepackage{csquotes}
\MakeOuterQuote{"}

\title{Rechnernetze Uni Bremen Fragekatalog}
\begin{document}
\maketitle
\section{Digitale Übertragung}
\begin{enumerate}
\item \textit{Was versteht man unter digitaler Übertragungstechnik?}\\
\-\hspace{.5cm}Antwort hier einfügen
\end{enumerate}
\section{Analoge Übertragung}

\section{POTS und Modemanschluss}

\section{Architekturmodelle -- OSI vs. Internet}

\section{Abschnittsicherungsschicht -- Konzepte}

\section{Abschnittsicherungsschicht -- LAPB}

\section{Netztypen- und Topologien}

\section{Klassisches Ethnernet vs. FDDI}

\section{Übertragungsmedien -- Kupfer vs. Glas}

\section{Lokale Netze -- Ethernet heute}

\section{Netzkopplung}

\section{IPv4 und IPv6}

\section{Rund um IP}

\section{Routing}

\section{Lokale Funknetze -- WLAN vs. Bluetooth}

\section{Heimanschlüsse und -netze}

\section{Transportschicht -- Überblick}

\section{Transportschicht -- UDP, TCP, SCTP}

\section{Kommunikationssteuerung}

\section{Binäre Kodierungsformate -- ASN.1, XDR, Protobuf etc.}

\section{Offene Dokumentverarbeitung}

\section{Textuelle Kodierungsformate -- SGML, XML, JSON etc.}

\section{Prozedurfernaufruf}

\section{Namensdienste -- X.500, LDAP und DNS}

\section{Betriebsprotokolle -- SNMP, dhcp und NTP}

\section{Die drei Säulen des Web -- URI, HTML, HTTP}

\section{Web -- Komponenten und Services}

\section{Klassische Internetanwendungen -- Telnet, FTP und NFS}

\section{E-Mail}

\section{Informationssicherheit}

\section{Internet of Things (IoT)}
\end{document}
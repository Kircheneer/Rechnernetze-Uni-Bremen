\documentclass[hidelinks]{article}
% Seitengröße und Ränder
\usepackage[a4paper]{geometry}
\usepackage[utf8]{inputenc}
\usepackage[ngerman]{babel}
\usepackage{graphicx}
\usepackage{hyperref}
\usepackage{wrapfig}
% Keine Einrückung am Paragraphenanfang
\usepackage{parskip}
% Anführungszeichen
\usepackage{csquotes}
\MakeOuterQuote{"}

\title{Rechnernetze Uni Bremen Fragekatalog}
\begin{document}
\maketitle
\section{Digitale Übertragung}
\begin{enumerate}
\item \textit{Was versteht man unter digitaler Übertragungstechnik?}

Die Technik, die, in Abhängigkeit von Bittakt, Symbolrate, Übertragungsmedium etc., digitale Signale von A nach B transportiert. %TODO 'etc.'

\item \textit{Skizziere einige Übertragungsverfahren zur digitale Übertragung von Bitfolgen.}

Siehe Manchestercode, NRZI etc. %TODO Zeichnung einfügen
\item \textit{Was versteht man unter dem Begriff Bitsynchronisation?}

Bei der Übertragung kommt es zu einer Verzögerung zwischen Absenden und Empfangen. Um ein korrektes Übertragen von digitale Signalen zu ermöglichen, muss gewährleistet sein, dass beide Parteien die selbe Taktrate verwenden. Die Synchronisation dieser Raten bezeichnet man als Bitsynchronisation.
\item \textit{Welche grundsätzlich verschiedenen Verfahren zur Taktgewinnung können unterschieden werden?}

\begin{itemize}
	\item Nutzen einer seperaten Taktleitung, die abwechselnd im Takt hi und low sendet 
	\begin{itemize}
		\item Teuer
		\item Die beiden Leitungen müssten exakt gleich lang sein
	\end{itemize}
	\item Kodieren von Bits über Flanken (z.B. Manchester-Code)
	\begin{itemize}
		\item Sehr hektisch
	\end{itemize}
	\item Verwenden von lokalen Uhren
	\begin{itemize}
		\item Regelmäßige Flanken müssen sichergestellt werden (z.B. durch NRZI-Code und/oder Synchronisationsbits)
	\end{itemize}
\end{itemize}
\item \textit{Welche verschiedenen Gleichlaufverfahren gibt es?}

\begin{itemize}
	\item Asynchroner Betrieb
	\begin{itemize}
		\item Takt für jedes Byte neu aufsetzen
		\item Unterscheidung zwischen Daten und "Pause" damit implizit		
	\end{itemize}	
	\newpage
	\item Synchroner Betrieb
	\begin{itemize}
		\item Keine implizite Unterscheidung zwischen Daten und "Pause"
	\end{itemize}
\end{itemize}
\end{enumerate}
\section{Analoge Übertragung}

\section{POTS und Modemanschluss}

\section{Architekturmodelle -- OSI vs. Internet}

\section{Abschnittsicherungsschicht -- Konzepte}

\section{Abschnittsicherungsschicht -- LAPB}

\section{Netztypen- und Topologien}

\section{Klassisches Ethnernet vs. FDDI}

\section{Übertragungsmedien -- Kupfer vs. Glas}

\section{Lokale Netze -- Ethernet heute}

\section{Netzkopplung}

\section{IPv4 und IPv6}

\section{Rund um IP}

\section{Routing}

\section{Lokale Funknetze -- WLAN vs. Bluetooth}

\section{Heimanschlüsse und -netze}

\section{Transportschicht -- Überblick}

\section{Transportschicht -- UDP, TCP, SCTP}

\section{Kommunikationssteuerung}

\section{Binäre Kodierungsformate -- ASN.1, XDR, Protobuf etc.}

\section{Offene Dokumentverarbeitung}

\section{Textuelle Kodierungsformate -- SGML, XML, JSON etc.}

\section{Prozedurfernaufruf}

\section{Namensdienste -- X.500, LDAP und DNS}

\section{Betriebsprotokolle -- SNMP, dhcp und NTP}

\section{Die drei Säulen des Web -- URI, HTML, HTTP}

\section{Web -- Komponenten und Services}

\section{Klassische Internetanwendungen -- Telnet, FTP und NFS}

\section{E-Mail}

\section{Informationssicherheit}

\section{Internet of Things (IoT)}
\end{document}